\chapter*{INTRODUCTION GÉNÉRALE}
\markboth{\MakeUppercase{INTRODUCTION GÉNÉRALE}}{}
%\addstarredchapter{INTRODUCTION GÉNÉRALES}
\addcontentsline{toc}{chapter}{INTRODUCTION GÉNÉRALE}
\adjustmtc
\thispagestyle{MyStyle}

Au Burkina Faso, comme dans de nombreux autres pays, l'accès à l'enseignement supérieur demeure un défi majeur pour de nombreux bacheliers après leurs études secondaires. Dans cette nation d'Afrique de l'ouest, les universités jouent un rôle crucial dans la formation des futurs leaders, professionnels et innovateurs \cite{unesco}. Cependant, malgré les aspirations et les efforts déployés par les étudiants, le taux d'échec persiste, constituant un obstacle majeur à la réalisation de leurs rêves académiques et professionnels \cite{lefaso}.

Cette réalité est également palpable dans les universités du Burkina Faso, où de nombreux étudiants se trouvent confrontés à des difficultés académiques qui compromettent leur réussite universitaire et leur avenir professionnel. Parmi ces institutions, l'Université Norbert Zongo se distingue en tant qu'établissement d'enseignement supérieur engagé dans l'excellence académique et le développement de ses étudiants. Cependant, même au sein de cette université renommée, le défi de la réussite académique demeure une préoccupation majeure.

Le constat d'un taux d'échec élevé parmi les étudiants universitaires au Burkina Faso, y compris à l'UNZ, soulève des questions cruciales sur les facteurs qui influent sur la réussite académique et sur les moyens de les aborder de manière proactive. Face à cette réalité, il devient impératif d'explorer de nouvelles approches et solutions innovantes pour soutenir les étudiants dans leur parcours éducatif et maximiser leurs chances de réussite.

C'est dans ce contexte que s'inscrit notre projet de recherche, qui est consacré sur la prédiction de réussite en licence économie à l'UNZ par l'intelligence artificielle. En reconnaissant la valeur prédictive des données disponibles, telles que les résultats académiques antérieurs, les années du Baccalauréat, et d'autres caractéristiques pertinentes, nous cherchons à fournir aux étudiants des informations précieuses qui leur permettront de mieux comprendre leurs chances de réussite et de prendre des décisions éclairées pour améliorer leur performance académique.

Le présent projet vise à élaborer une application web intégrant des techniques d'intelligence artificielle, dans le but de prédire de manière précise la réussite en licence des étudiants spécialisés en économie. Cette application utilisera des méthodes d'analyse avancées pour évaluer les performances académiques des étudiants, en se basant sur diverses caractéristiques et indicateurs pertinents. En fournissant des prédictions fiables et des conseils personnalisés, cette solution technologique s'efforcera d'aider les étudiants à mieux comprendre leur potentiel académique et à prendre des décisions éclairées pour optimiser leur parcours universitaire. Par cette approche proactive, le développement de cette application aspire à contribuer significativement à la réduction du taux d'échec et à la promotion de la réussite universitaire, tant au sein de l'établissement qu'au niveau plus large de l'enseignement supérieur.

Pour atteindre notre objectif, nous allons d'abord fournir un aperçu de l'intelligence artificielle, de ses domaines et de son évolution, afin de contextualiser notre projet. Ensuite, nous explorerons les bases du machine learning, en mettant en lumière son rôle essentiel dans la mise en œuvre de solutions d'intelligence artificielle. Nous aborderons également l'état de l'art dans le domaine de la prédiction de la réussite des étudiants, ainsi que notre approche spécifique pour répondre à ce défi. Enfin, nous présenterons les aspects pratiques de notre projet, de l'implémentation à la discussion des résultats obtenus, en passant par les perspectives futures.