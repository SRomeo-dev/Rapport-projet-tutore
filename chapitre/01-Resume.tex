\newpage
\thispagestyle{empty}
\begin{center}
  \renewcommand*{\familydefault}{\defaultFont}
  \fontsize{12pt}{12pt}\selectfont%
  \textbf{
  Prédiction de réussite de licence économie à l'UNZ par l'intelligence artificielle\\%
  }
\vspace{15pt} {%
  \begin{spacing}{0.05}
    \rule{200pt}{2pt}\\
    \rule{200pt}{0.75pt}\\
  \end{spacing}
  \renewcommand*{\familydefault}{\defaultFont}
  \fontsize{14pt}{14pt}\selectfont%
  \vspace{15pt}
  \textbf{Paroguenssaongo Roméo SAWADOGO}
  \vspace{8pt}
  \begin{spacing}{0.05}
    \rule{200pt}{0.75pt}\\
    \rule{200pt}{2pt}\\
  \end{spacing}
}
\end{center}

%Français
\fontsize{12pt}{12pt}\selectfont%
\underline{\textbf{Résumé:}}\\
La prédiction de la réussite académique des étudiants est un enjeu crucial en Sciences Économiques et de Gestion (SEG). Cette recherche vise à développer un modèle prédictif pour identifier les étudiants susceptibles de réussir leur licence économie dès la première année, en utilisant des techniques avancées d’intelligence artificielle et de machine learning sur des données comme les notes et l'année d'obtention du BAC. Après la collecte et le prétraitement des données, divers algorithmes de machine learning ont été entraînés et évalués via des métriques telles que la précision, le rappel, et le F1-score. Certains modèles se sont révélés particulièrement efficaces pour identifier les étudiants à risque d'échec. Enfin, une application web a été développée pour permettre une utilisation pratique du modèle prédictif.
\\
\begin{spacing}{1}
\underline{\textbf{Mots clés:}} Intelligence artificielle – Machine learning – Prédiction académique –  Économie – Enseignement supérieur.\\
\end{spacing}
\vspace*{2cm}
\underline{\textbf{Abstract :}}\\
Predicting the academic success of students is a crucial issue in Economics and Management Sciences (SEG). This research aims to develop a predictive model to identify students likely to succeed in their economics degree from the first year, using advanced artificial intelligence and machine learning techniques on data such as grades and year of obtaining the degree. BAC. After data collection and preprocessing, various machine learning algorithms were trained and evaluated via metrics such as precision, recall, and F1-score. Some models have proven particularly effective in identifying students at risk of failure. Finally, a web application was developed to allow practical use of the predictive model.
\\
\par
\begin{spacing}{1}
\underline{\textbf{Key-words:}} Artificial intelligence – Machine learning – Academic prediction – Economy – Higher education.\\
\end{spacing}
\clearpage


    