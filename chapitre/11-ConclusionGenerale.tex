\chapter*{CONCLUSION GÉNÉRALE}
\markboth{\MakeUppercase{CONCLUSION GÉNÉRALE}}{}
\addcontentsline{toc}{chapter}{CONCLUSION GÉNÉRALE}
\adjustmtc
\thispagestyle{MyStyle}

Le présent projet de recherche s'est attaché à explorer l'utilisation de l'intelligence artificielle comme outil de prédiction de la réussite des étudiants en licence, avec un accent particulier sur le domaine des Sciences Économiques et de Gestion à l'Université Norbert Zongo. À travers notre étude, nous avons cherché à répondre à un besoin pressant au sein de l'enseignement supérieure au Burkina Faso : celui d'identifier les étudiants en difficulté dès leur entrée à l'université et de leur fournir un soutien approprié pour maximiser leurs chances de réussite.
En analysant en profondeur l'efficacité de l'intelligence artificielle dans la prédiction de la réussite académique des étudiants en économie, nous mettons en lumière l'importance de cette approche innovante pour soutenir leur parcours académique. Notre recherche offre des perspectives précieuses pour les décideurs politiques, les administrateurs universitaires et les praticiens de l'enseignement supérieur, en identifiant les facteurs prédictifs de la réussite académique et en proposant des pistes d'action pour améliorer les résultats des étudiants.