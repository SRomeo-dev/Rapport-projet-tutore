\chapter*{PRÉAMBULE}
\addstarredchapter{PRÉAMBULE}
%\addcontentsline{toc}{chapter}{REMERCIEMENT}
%\adjustmtc
\thispagestyle{MyStyle}

L'Université Norbert Zongo (UNZ), anciennement l'Université de Koudougou, a été créée par décret N° 2005-460/PRES/PM/ MESSRS/MFB du 31 août 2005, résultant de la transformation de l'École Normale Supérieure de Koudougou (ENSK). Son appellation actuelle, décidée lors du Conseil des Ministres du 21 juillet 2017, rend hommage à Norbert Zongo, un journaliste émérite dont l'engagement pour la bonne gouvernance demeure exemplaire. L'UNZ est implantée au bord de l'Avenue Maurice Yaméogo (Route Nationale N°14) à Koudougou, chef-lieu de la province du Boulkiemdé et de la région du Centre-Ouest. Sa devise, "Scientia Excelle Ut Melius Servias", reflète son engagement pour l'excellence académique et le service à la communauté.

Conformément aux statuts approuvés par le décret N° 2017-0144 PRES/ PM/MESRSI/MINEFID du 22 mars 2017, l'UNZ est un Établissement Public de l'État à caractère Scientifique, Culturel et Technique, jouissant d'une personnalité morale et d'une autonomie scientifique, pédagogique, administrative et financière. Ses structures administratives, techniques et de gestion comprennent le Conseil d'Administration, le Conseil de la Formation et de la Vie Universitaire, le Conseil Scientifique, la Présidence, ainsi que les établissements d'enseignement, de formation et les centres de recherche.\\
Elle dispose en son sein de sept établissements, à savoir :
\begin{itemize}
	\item[\ding{118}] l'UFR-SEG;
	\item[\ding{118}] l'UFR-LSH;
	\item[\ding{118}] l'UFR-ST;
	\item[\ding{118}] l'IUT;
	\item[\ding{118}] l'ED-ST;
	\item[\ding{118}] l'ED-LACOSHS;
	\item[\ding{118}] le CPU
\end{itemize}
Parmi les sept établissements abrités par l'UNZ, nous avons eu le privilège de suivre notre formation à l'UFR-ST, créée en 2014. Cette unité s'inscrit dans le cadre du système LMD et offre une gamme variée de formations dans les domaines scientifiques et technologiques.\\
L’UFR-ST est un établissement supérieur d’enseignement général dont l’objectif est de former les étudiants dans les domaines scientifiques et technologiques. Dans le cadre du système LMD, elle prépare les étudiants à des Licences, des Masters et des Doctorats dans les filières suivantes :
\begin{itemize}
   \item[\ding{118}] MPCI : ce parcours aboutit aux licences en Mathématique, Physique, Chimie et Informatique, puis au Master et au Doctorat.
    
    \item[\ding{118}] SVT : ce parcours aboutit aux licences en Biologie Générale et Biochimie Fondamentale, puis au Master et au Doctorat.
\end{itemize}
L'accès à ces différentes filières se fait à travers des orientations effectuées via la plate-forme Campus Faso. Pour le cycle Master et le cycle Doctorat, la sélection se fait également via cette plateforme, sur la base d'un dossier de candidature.\\
Le cycle LMD favorise également la réalisation de projets tutorés dans la spécialité informatique, offrant aux étudiants en fin de cycle de licence une expérience pratique et leur permettant de développer leurs compétences.
Dans cette optique, nous avons suivi un stage de trois mois à la DSI relevant de la Vice-présidence chargée des Enseignements et de l'Innovation Pédagogique de l'UNZ, du 01 er Mars au 31 mai 2024, dans le cadre de notre formation en Licence Informatique. Ce stage, intégré à notre cursus académique, vise à nous familiariser avec le milieu professionnel et à enrichir nos connaissances pratiques, contribuant ainsi à la réussite totale de notre projet tutoré.